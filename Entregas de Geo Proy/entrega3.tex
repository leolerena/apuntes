\documentclass[10pt,a4paper]{article}
\usepackage[latin1]{inputenc}
\usepackage{amsmath}
\usepackage{amsfonts}
\usepackage{amssymb}
\usepackage{graphicx}
\usepackage{amsmath,amsthm,amssymb}
% Include links
\usepackage{hyperref}
\usepackage{color}
\usepackage{mathtools}
\usepackage{pifont}
\usepackage{tikz-cd}
\usepackage{tcolorbox}
\usepackage{mdframed}
\usepackage{tcolorbox}
\usepackage{fancyhdr}
\pagestyle{fancy}
\lhead{Geometr�a Proyectiva}
\author{Leopoldo Lerena}


\newenvironment{exercise}[2][Ejercicio]{\begin{trivlist}
		\item[\hskip \labelsep {\bfseries #1}\hskip \labelsep {\bfseries #2.}]}{\end{trivlist}}
\newenvironment{reflection}[2][Resoluci\']{\begin{trivlist}
		\item[\hskip \labelsep {\bfseries #1}\hskip \labelsep {\bfseries #2.}]}{\end{trivlist}}


\newcommand{\norm}[1]{\left\lVert#1\right\rVert}
\newcommand{\ti}{\tilde}
\newcommand{\R}{\mathbb R}
\newcommand{\Proy}{\mathbb P ^n (k)}
\newcommand{\A}{\mathbb A}


\title{Tercer entrega.}
\date{Octubre 2018}

\begin{document}
	\maketitle
	
	
	\textbf{Plano rectificante.}
	
	Sea $\alpha:I \to \R^3$ una curva regular con curvatura y torsi�n nunca nulas. Denotamos $\{T,N,B\}$ a su triedro de Frenet y el \textit{plano rectificante} es el plano con normal $N$ que pasa por $\alpha$. 
	
	Supongamos que todos los planos rectificantes de $\alpha$ est�n a la misma distancia $c \geq 0$ del origen. Probar que, en los puntos $t \in I$ donde $\left(\dfrac{ \kappa}{\tau}\right)'$ y donde $\langle T(t),\alpha(t) \rangle$ son distintos de 0 vale la siguiente igualdad
	\[ ||\alpha(t)||^2 = \left(\dfrac{- \tau c - \dfrac{\kappa^2 c}{\tau}  - \dfrac{\kappa}{\tau}}{\left(\dfrac{\kappa}{\tau}\right)'}\right)^2 + c^2 + \left(\dfrac{-\tau c - \dfrac{\kappa^3 c}{\tau^2}  - \dfrac{\kappa^2}{\tau^2}}{\left(\dfrac{\kappa}{\tau}\right)'}\right)^2 .\]
	
	
	\bigskip
	
	\textit{Soluci�n.}
	
	Lo primero que voy a hacer es reparametrizar la curva por longitud de arco para eventualmente poder usar Frenet-Serret m�s comodamente. Claramente este resultado no depende de la parametrizaci�n de la curva sino m�s bien de su traza por lo tanto voy a obtener el mismo resultado. Como tanto la curvatura y la torsi�n son no nulas tengo un triedro de Frenet-Serret.
	
	Si llamo al plano rectificante a $\alpha$ en el momento $t$ a $R(t) = \langle N(t),\alpha(t) \rangle $, donde $N(t)$ es el vector del triedro de Frenet-Serret. Que su distancia al origen sea constante, es decir que $d(R(t),0) = c$ para alg�n $c \in \R_{>0}$ es lo mismo que \footnote{Ac� no me queda claro porqu� necesariamente es igual a $c$ y no a $-c$, en tal caso los signos me quedan distintos a los pedidos.}
	\[\langle N(t),\alpha(t) \rangle = c , \hspace{0.75cm} \text{para todo t} \in \R.\]
	
	Dado que el triedro de Frenet-Serret es una base ortonormal de $\R^3$ puedo escribir $\alpha(t) = \langle T(t),\alpha(t) \rangle T(t) + \langle N(t),\alpha(t) \rangle N(t) + \langle B(t),\alpha(t) \rangle B(t)$ entonces para calcular su norma basta con tomar el producto interno contra s� mismo y dado que es una base ortonormal lo �nico que sobrevive es
	\[ ||\alpha(t)||^2 = \langle T(t),\alpha(t) \rangle^2 + \langle N(t),\alpha(t) \rangle^2 + \langle B(t),\alpha(t) \rangle^2 .\]
	Por lo tanto me alcanza con calcular $\langle T(t),\alpha(t) \rangle$ y $\langle B(t),\alpha(t) \rangle$. Para eso uso que la distancia al origen de la recta normal es constante y derivando esta igualdad obtengo lo siguiente. Fijado $t$.
	\begin{align*}
	\langle N,\alpha \rangle' &= (c)' \\
	\langle N',\alpha \rangle + \langle N,T \rangle &= 0 .\\
	\end{align*}
	Ac� utilizo Frenet-Serret para reescribir $N'$ en t�rminos del triedro y que $N$ es ortogonal a $T$.
		
	\begin{align*}
	\langle -\tau B -\kappa T ,\alpha(t) \rangle  &= 0   \\
	\langle -\tau B, \alpha \rangle   &=  \langle \kappa T ,\alpha \rangle      \\
	-\tau \langle  B, \alpha \rangle   &= \kappa \langle  T ,\alpha \rangle      \\
	\end{align*}
	Con lo cual llego a la siguiente ecuaci�n que me relaciona los coeficientes que busco. 
	\begin{equation}\label{igualdad}
	 \langle  B, \alpha \rangle   = - \dfrac{ \kappa}{\tau} \langle  T ,\alpha \rangle.
	\end{equation}
	
	
	
	Entonces derivando esta igualdad obtengo lo siguiente.
	
	\begin{equation*}
	\langle  B', \alpha \rangle + \langle  B, T \rangle   = - \left( \dfrac{ \kappa}{\tau}\right)'  \langle  T ,\alpha \rangle + \dfrac{ \kappa}{\tau} \langle  T' ,\alpha \rangle + \dfrac{ \kappa}{\tau} \langle  T ,T \rangle.
	\end{equation*}
	 Reacomodando y expandiendo con Frenet-Serret y usando que el triedro es una base ortonormal llego a la igualdad que buscaba. Ac� utilizo que $\left(\dfrac{ \kappa}{\tau}\right)' \neq 0$ para poder dividir por eso.
	 
	 \begin{align*}
	 - \tau c   &= - \left( \dfrac{ \kappa}{\tau}\right)'  \langle  T ,\alpha \rangle - \dfrac{ \kappa^2 c}{\tau}  - \dfrac{ \kappa}{\tau}. \\
	  \left( \dfrac{ \kappa}{\tau}\right)'  \langle  T ,\alpha \rangle &= - \tau c - \dfrac{ \kappa^2 c}{\tau}  - \dfrac{ \kappa}{\tau}. \\
	   \langle  T ,\alpha \rangle &= \dfrac{- \tau c - \dfrac{ \kappa^2 c}{\tau}  - \dfrac{ \kappa}{\tau}}{\left(\dfrac{ \kappa}{\tau}\right)'}
	 \end{align*}
	Volviendo a \ref{igualdad} puedo reemplazar el valor hallado para llegar a que
	\[ \langle  B, \alpha \rangle = \dfrac{- \tau c - \dfrac{ \kappa^3 c}{\tau^2}  - \dfrac{ \kappa^2}{\tau^2}}{\left(\dfrac{ \kappa}{\tau}\right)'}. \]
	Juntando todo, esto me dice que
	\[ ||\alpha(t)||^2 = \left(\dfrac{-\tau c - \dfrac{\kappa^2 c}{\tau}  - \dfrac{\kappa}{\tau}}{\left(\dfrac{\kappa}{\tau}\right)'}\right)^2 + c^2 + \left(\dfrac{-\tau c - \dfrac{\kappa^3 c}{\tau^2}  - \dfrac{\kappa^2}{\tau^2}}{\left(\dfrac{\kappa}{\tau}\right)'}\right)^2 .\]
	
	\qed
	
	
\end{document}	