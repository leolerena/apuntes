


\newcommand{\norm}[1]{\left\lVert#1\right\rVert}
\newcommand{\ti}{\tilde}

\title{Primer entrega.}
\author{Leopoldo Lerena }
\date{Septiembre 2018}

\begin{document}

\maketitle

\section*{Ejercicio 1}
\subsection*{(a)}
\textit{Sean $U$ un espacio vectorial sobre $\mathbb R$, $b$ una forma bilineal sim\'etrica en $U$ y $Q$ la forma
cuadr\'atica asociada a $b$. Decimos que $U$ es un espacio hiperb\'olico si $b$ es no degenerada
y si signatura tiene igual cantidad de $1$ que de $-1$. En particular, esto implica que
dim($U$) $= 2p$ y que la signatura de $b$ es $(2p, p)$. Si la dimensi\'on de $U$ es $2$, decimos que U es un plano hiperbólico.}


\bigskip

Quiero descomponer un espacio hiperbólico en una suma de planos hiperbólicos ortogonales entre sí. Acá voy a usar que $b$ es una forma bilineal simétrica  no degenerada. Por lo tanto  existe una base ortonormal $\mathcal{B} = \{ x_1, \dots, x_{2p}\}$ de $\mathcal U$ tal que  la matriz asociada a $b$  es de la siguiente manera.
\[
[b]_{\mathcal{B}}
=
\begin{bmatrix}
1 & 0  & \dots & 0 & 0 \\
0 & 1  & \dots & 0 & 0 \\
\vdots & \vdots  & \ddots & \vdots \\
0 & 0  & \dots & -1 & 0 \\
0 & 0  & \dots & 0 & -1
\end{bmatrix}
.\]

Como los autovectores son ortogonales entre sí, los espacios generados por ellos mismos también van a ser ortogonales entre sí y de esa manera descomponen al espacio hiperbólico en una suma ortogonal.

\[ \mathcal{U} = \langle x_1 \rangle \oplus \dots \oplus \langle x_{2p} \rangle .\]

 Afirmo que sabiendo esto puedo juntar de a pares los espacios generados por los autovectores, ya que la cantidad de autovectores es $2p$. Los pares los elijo de manera que un autovector le corresponda el autovalor $+1$ y al otro autovector le corresponda el autovalor $-1$. Tomo como candidato $H_{i} = \langle x_i \rangle \oplus \langle x_{i+p} \rangle $. Veamos que éste es un plano hiperbólico. 
 
 Para eso tendría que ver que la forma bilineal $b$ al restringirse es no degenerada y en la signatura tiene la misma cantidad de $+1$ que de $-1$. 
 
 \begin{equation*}
 	\left. [b]\right| _ {H_i}	=
 	\begin{bmatrix}
 	b(x_{i}, x_{i}) & \dfrac{b(x_{p+i}, x_{i})}{2}  \\
 	\dfrac{b(x_{p+i}, x_{i})}{2} & b(x_{p+i}, x_{p+i}) \\
 	\end{bmatrix}.
\end{equation*} 
  Y dado que por cómo los elegí antes de manera que $b(v,w)=0$ para todo $v \in H_i$,$w \in H_j$ con $j \neq i$, la matriz asociada resulta ser
  \begin{equation*}
  \left. [b]\right| _ {H_i}	=
  \begin{bmatrix}
  1 & 0  \\
  0 & -1 \\
  \end{bmatrix}.
  \end{equation*}
Tenemos finalmente una descomposición de $\mathcal U$ en suma directa de planos hiperbólicos de la siguiente manera
\begin{equation*}
 \mathcal{U} = H_1 \oplus \dots \oplus H_{p}. 
\end{equation*} 


\subsection*{(b)}

\textit{Un vector $v$ en $U$ se dice isotr\'opico si es ortogonal a si mismo, es decir, si $b(v, v) = 0$.
Probar que en un plano hiperb\'olico los vectores isotr\'opicos son la uni\'on de dos
rectas que pasan por el origen.}

\bigskip

Queremos caracterizar a los vectores isotrópicos de los planos hiperbólicos. Sea $H$ un plano hiperbólico. Considero la base $\mathcal B = \{v_1 , v_2\}$ que hace que la matriz asociada a la forma bilineal $b$ sea de la siguiente forma
 \begin{equation*}
[b]_\mathcal{B}	=
\begin{bmatrix}
1 & 0  \\
0 & -1 \\
\end{bmatrix}.
\end{equation*} 

Un vector  $v \in H$ genérico se escribe como $v=rv_1 + sv_2$. Por definición de isotrópico cumple que 
\[ b(v,v) = 0 \hspace{0.25cm} \text{sí y solo si} \hspace{0.25cm} v^{t}_{\mathcal{B}}[b]_{\mathcal{B}}v_{\mathcal{B}}=0. \]
Expandiendo esto es lo mismo que ver que
\[ r^2 - s^2 = 0. \]
Esto dice que los módulos deben ser iguales, es decir que $|r|=|s|$. Nos definidas dos rectas $L_1, L_2$ que son
\begin{align*}
L_1 &= \{u \in \mathcal U : u =\lambda(v_1 + v_2) , \lambda \in \mathbb{R}\} \\
L_2 &= \{u \in \mathcal U : u =\lambda(v_1 - v_2) , \lambda \in \mathbb{R}\}. \\
\end{align*}




\subsection*{(c)}
\textit{Un subespacio $S$ se dice isotr\'opico si est\'a formado por vectores isotr\'opicos. Probar
	que en todo espacio hiperb\'olico de dimensi\'on $2p$ existen subespacios isotr\'opicos de
	dimensi\'on $p$.}

\bigskip 

Buscamos un subespacio de dimensión $p$ tal que esté formado por vectores isotrópicos. Para eso voy a construir explicitamente este subespacio usando del resultado anterior  que puedo descomponer al espacio como una suma directa de planos hiperbólicos. 
\begin{equation*}
\mathcal{U} = H_1 \oplus \dots \oplus H_{p} 
\end{equation*}
De cada plano hiperbólico puedo obtener una recta isotrópica que es un subespacio vectorial de dimensión 1. Si tomo la suma de todas estas rectas obtengo un subespacio de $\mathcal{U}$ de dimensión $p$. Tendría que chequear que realmente es un subespacio de vectores isotrópicos (dado que claramente es un subespacio). Para eso si defino como $\langle v_i \rangle \subset H_i$ la recta de vectores isotrópicos dentro del i-ésimo espacio hiperbólico. Lo que afirmaba antes es que el siguiente subespacio $S$ cumple lo pedido.
\[S \coloneqq \left\langle v_1 \right\rangle \oplus \dots \oplus \left\langle v_p \right\rangle   \]
Sea $v= \sum_{i=1}^{p} \lambda_{i}v_{i}$ un vector genérico del subespacio $S$ en la escritura de la suma directa. Veamos que $b(v,v)=0$.

\begin{align*}
	b(v,v) &= b\left( \sum_{i=1}^{p} \lambda_{i}v_{i}, \sum_{i=1}^{p} \lambda_{i}v_{i}\right) \\
	& = \sum_{i=1}^{p} \lambda_{i}^{2} b(v_i,v_i) + \sum_{i<j, i=1}^{j=p} \lambda_{i}\lambda_{j} b(v_j,v_i) \\
	& = 0
\end{align*}
Porque los vectores $v_i$ son isotrópicos y son ortogonales con los $v_j$  por estar en distintos planos hiperbólicos que ya son ortogonales entre sí. Por lo tanto el subespacio $S$ es un subespacio isotrópico de dimensión $p$.

\bigskip


\section*{Ejercicio 2}
\textit{Determinar la ecuación normal euclídea de la siguiente cu\'adrica en $\mathbb R^3$, especificando la transformaci\'on ortogonal af\'in que act\'ua sobre ella para obtenerla.}
\begin{align}
x_1^2 + 2x_1x_2 + 2x_1x_3 + 6x_1 + 2x_2 + 8x_3 - 1 = 0.
\end{align}

\bigskip

Busco la ecuación normal euclídea de la siguiente cuádrica de $\mathbb R ^3$. 
\[ F(x) = x_1 ^2 +2x_1 x_2 +2x_1 x_3 +6x_1 +2x_2 +8x_3 -1  = 0\]
La matriz asociada a la forma bilineal $\beta_f$ es la siguiente.
 \[ \beta_f =
 \begin{bmatrix}
 1 & 1 & 1  \\
 1 & 0 & 0 \\
 1 & 0 & 0
 \end{bmatrix}.
 \]
Tal que si calculo sus autovalores factorizando el polinomio característico de la matriz $XI - \beta_f$ y los autovectores calculando el núcleo de $\lambda I - \beta_f$, donde $\lambda$ es un autovalor,  obtengo los siguientes resultados 
\begin{itemize}
	\item $\lambda_1=2$, $v_1= \dfrac{1}{\sqrt{6}}(2,1,1)$ 
	\item $\lambda_2=-1 $, $v_2= \dfrac{1}{\sqrt{3}}(-1,1,1)$
	\item $\lambda_3=0$, $v_3= \dfrac{1}{\sqrt{2}}(0,1,-1)$.
\end{itemize}

Llamaré $O$ a la matriz ortonormal que tiene como columnas a los autovectores de la forma bilineal y $D$ la matriz diagonal que tiene los autovalores $\lambda_i$. Entonces la forma cuadrática la puedo escribir de la siguiente manera.
\begin{equation}
\label{formula}
 F(x) = x^{t}ODO^{t}x + b^{t}OO^{t}x -1 
 \end{equation}
Calculo $\ti x = O^{t}x$ que me va a dar mis nuevas coordenadas. En este caso me queda,
\[
O^tx = 
\begin{bmatrix}
&\dfrac{1}{\sqrt{6}}\left( 2x_1 + x_2 + x_3\right)  \\
&\dfrac{1}{\sqrt{3}}\left( -x_1 + x_2 + x_3 \right)  \\
&\dfrac{1}{\sqrt{2}} \left( x_2 - x_3\right)  \\
\end{bmatrix}
\]




En este caso  la parte lineal de la cuádrica nos queda lo siguiente.
\[b^tO(\tilde{x_1},\tilde{x_2},\tilde{x_3})^t = \left( \dfrac{22}{\sqrt{6}},\dfrac{4}{\sqrt{3}},\dfrac{6}{\sqrt{2}}\right)(\tilde{x_1},\tilde{x_2},\tilde{x_3})^t \]
Desarrollando y completando cuadrados nos queda la expresión
\begin{align*}
F(x) = x^{t}ODO^{t}x + b^{t}OO^{t}x -1 &= 2x_1^2 - x_2^2 + \frac{22}{\sqrt{6}}x_1 - \frac{4}{\sqrt{3}}x_2 - \frac{6}{\sqrt{2}}x_3 - 1 = \\
 &= (\sqrt{2}x_1 + \frac{11}{2\sqrt{3}})^2 -(x_2 + \frac{2}{\sqrt{3}})^2 -3\sqrt{2}(x_3 + \frac{13}{4\sqrt{2}})
\end{align*}
  En este caso se corresponde a una $C_{(2,-1,0), 3\sqrt2}$ .
 \[ F \sim 2x_1 ^2 -  x_2 ^2 - 3\sqrt2 x_3 \]
 
 Definimos la transformación ortogonal afín $f$ que manda la base que hace que la cuádrica esté en su forma canónica a la base estándar de $\mathbb R ^3$
 de la siguiente manera.
  \begin{align*}
   f\left( \tilde{x_1} \right) &= \dfrac{1}{\sqrt{6}}\left( 2x_1 + x_2 + x_3\right) - \dfrac{11}{\sqrt{6}}  \\
   f\left( \tilde{x_2} \right) &= \dfrac{1}{\sqrt{3}}\left( -x_1 + x_2 + x_3\right) - \dfrac{2}{\sqrt{3}}  \\
   f\left( \tilde{x_3} \right)  &= \dfrac{1}{\sqrt{2}} \left( x_2 - x_3 - \frac{13}{4\sqrt{2}} \right)  \\
 \end{align*}

 Ésta nos va a mandar la base canónica de $\mathbb R ^3$ en la base tal que la cuádrica tiene la forma canónica. Esta transformación afín ortogonal $f$ es la que hace que nos quede 
\[ G(x) \coloneqq F(f(x)) \sim 2x_1 ^2 -  x_2 ^2 + x_3 \]


\end{document}
