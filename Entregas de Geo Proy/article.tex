\documentclass{article}
\usepackage[utf8]{inputenc}
\usepackage{amsfonts}
\usepackage{amsmath}
\usepackage{amssymb}
\usepackage[margin=1.25in]{geometry}
\usepackage{hyperref}
\usepackage{color}
\usepackage{mathtools}
\usepackage{pifont}
\usepackage{fancyhdr}
\pagestyle{fancy}
\lhead{Geometría Proyectiva}


\newcommand{\norm}[1]{\left\lVert#1\right\rVert}
\newcommand{\ti}{\tilde}

\title{Primer entrega.}
\author{Leopoldo Lerena }
\date{Septiembre 2018}

\begin{document}

\maketitle

\section*{Ejercicio 1}
\subsection*{(a)}
Quiero descomponer un espacio hiperbólico en una suma de planos hiperbólicos ortogonales entre sí. Acá voy a usar que $b$ es una forma bilineal simétrica. Por lo tanto por el teorema espectral sé que existe una base ortonormal $\mathcal{B} = \{ x_1, \dots, x_{2p}\}$ tal que  diagonaliza a $b$ de la siguiente manera.
\[
[b]_{\mathcal{B}}
=
\begin{bmatrix}
1 & 0  & \dots & 0 & 0 \\
0 & 1  & \dots & 0 & 0 \\
\vdots & \vdots  & \ddots & \vdots \\
0 & 0  & \dots & -1 & 0 \\
0 & 0  & \dots & 0 & -1
\end{bmatrix}
\]

Como los autovectores se pueden elegir de manera que sean ortogonales entre sí, los espacios generados por ellos mismos también van a ser ortogonales entre sí y de esa manera descomponen al espacio hiperbólico en una suma ortogonal.

\[ \mathcal{U} = \langle x_1 \rangle \oplus \dots \oplus \langle x_{2p} \rangle \]

 Afirmo que sabiendo esto puedo juntar de a pares los espacios generados por los autovectores, ya que la cantidad de autovectores es $2p$. Los pares los elijo de manera que un autovector le corresponda el autovalor $+1$ y al otro autovector le corresponda el autovalor $-1$. Tomo como candidato $H_{i} = \langle x_i \rangle \oplus \langle x_{i+p} \rangle $. Veamos que éste es un plano hiperbólico. 
 
 Para eso tendría que ver que la forma bilineal $b$ al restringirse es no degenerada y en la signatura tiene la misma cantidad de $+1$ que de $-1$. Pero esto es claro por como elegí a este subespacio. Si considero la matriz con respecto a la base dada por los generadores $x_i$ y $x_{i+p}$
 
 \begin{equation*}
 	\left. [b]\right| _ {H_i}	=
 	\begin{bmatrix}
 	b(x_{i}, x_{i}) & \dfrac{b(x_{p+i}, x_{i})}{2}  \\
 	\dfrac{b(x_{p+i}, x_{i})}{2} & b(x_{p+i}, x_{p+i}) \\
 	\end{bmatrix}
\end{equation*} 
  Y por ser los autovectores de una base ortogonal, lo que obtengo es que la matriz que me queda es la siguiente.
  \begin{equation*}
  \left. [b]\right| _ {H_i}	=
  \begin{bmatrix}
  1 & 0  \\
  0 & -1 \\
  \end{bmatrix}
  \end{equation*}
Se puede ver en esta matriz que la forma bilineal es simétrica y no degenerada dado que 0 no aparece en la diagonal. De manera similar su signatura tiene la misma cantidad de $+1$ que de $-1$. Esto me dice que es un plano hiperbólico. Por lo tanto basta con tomar la siguiente suma de planos hiperbólicos.
\begin{equation*}
 \mathcal{U} = H_1 \oplus \dots \oplus H_{p} 
\end{equation*} 


\subsection*{(b)}

Queremos caracterizar a los vectores isotrópicos de los planos hiperbólicos. Dado que es un espacio vectorial de dimensión 2 existe una base \textit{canónica} que llamaremos $E$.
Para eso voy a volver a usar el teorema espectral para encontrarme la base que diagonaliza
 \begin{equation*}
D \coloneqq [b]_\mathcal{B}	=
\begin{bmatrix}
1 & 0  \\
0 & -1 \\
\end{bmatrix}
\end{equation*} 

Sea $S$ la matriz de autovectores que diagonaliza a la forma bilineal. Lo que tenemos es que otra manera de decir que un vector $v$ es isotrópico es la siguiente.
\[ b(v,v) = 0 \iff v^{t}S^{t}DSv=0 \]
Sea $w = (x, y)$ un vector genérico, veamos cuándo se anula en la forma bilineal. 
\[ w^{t}Dw = 0 \iff x^{2} - y^{2} = 0 \]
Y esto sucede cuando $x=y$ o bien $x=-y$. Es decir $w \in \langle (1,1) \rangle $ y $w \in \langle (-1,1) \rangle$ Esto nos define ambas rectas pero recordemos que estamos pensandolo en la base ortonormal. Por lo tanto para despejar el vector $v$ tengo que hacer lo siguiente.
\[ Sv = w \implies v= S^{t}w\]
Por lo tanto las rectas que me quedan definidas son las siguientes dos.

\begin{equation*}
L_1	\coloneqq
\left\langle  S^t\begin{bmatrix}
1   \\
1  \\
\end{bmatrix}
\right\rangle 
\end{equation*}
\begin{equation*}
L_2	\coloneqq
\left\langle  S^t\begin{bmatrix}
1   \\
-1  \\
\end{bmatrix}
\right\rangle 
\end{equation*} 

\subsection*{(c)}
Buscamos un subespacio de dimensión $p$ tal que esté formado por vectores isotrópicos. Para eso voy a construir explicitamente este subespacio usando del resultado anterior  que puedo descomponer al espacio como una suma directa de planos hiperbólicos. 
\begin{equation*}
\mathcal{U} = H_1 \oplus \dots \oplus H_{p} 
\end{equation*}
De cada plano hiperbólico puedo obtener una recta isotrópica que es un subespacio vectorial de dimensión 1. Si tomo la suma de todas estas rectas obtengo un subespacio de $\mathcal{U}$ de dimensión $p$. Tendría que chequear que realmente es un subespacio de vectores isotrópicos (dado que claramente es un subespacio). Para eso si defino como $\langle v_i \rangle \subset H_i$ la recta de vectores isotrópicos dentro del i-ésimo espacio hiperbólico. Lo que afirmaba antes es que el siguiente subespacio $S$ cumple lo pedido.
\[S \coloneqq \left\langle v_1 \right\rangle \oplus \dots \oplus \left\langle v_p \right\rangle   \]
Sea $v= \sum_{i=1}^{p} \lambda_{i}v_{i}$ un vector genérico del subespacio $S$ en la escritura de la suma directa. Veamos que $b(v,v)=0$.

\begin{align*}
	b(v,v) &= b\left( \sum_{1}^{p} \lambda_{i=i}v_{i}, \sum_{1}^{p} \lambda_{i=i}v_{i}\right) \\
	& = \sum_{i=1}^{p} \lambda_{i}^{2} b(v_i,v_i) + \sum_{i<j, i=1}^{j=p} \lambda_{i}\lambda_{j} b(v_j,v_i) \\
	& = 0
\end{align*}
Porque los vectores $v_i$ son isotrópicos y son ortogonales con los $v_j$  por estar en distintos planos hiperbólicos que ya son ortogonales entre sí. Por lo tanto el subespacio $S$ es un subespacio isotrópico de dimensión $p$.

\bigskip


\section*{Ejercicio 2}
Busco la ecuación normal euclídea de la siguiente cuádrica de $\mathbb R ^3$. 
\[ F(x) = x_1 ^2 +2x_1 x_3 +6x_1 +2x_2 +8x_3 -1  = 0\]
La matriz asociada a la forma bilineal $\beta_f$ es la siguiente.
 \[ \beta_f =
 \begin{bmatrix}
 1 & 0 & 1  \\
 0 & 0 & 0 \\
 1 & 0 & 0
 \end{bmatrix}
 \]
Tal que si calculo sus autovalores y autovectores obtengo lo siguiente. 
\begin{itemize}
	\item $\lambda_1=\frac{1}{2} (1+\sqrt{5})$, $v_1= (\frac{1}{2}(1+\sqrt{5}),0,1)$ 
	\item $\lambda_2=\frac{1}{2} (1-\sqrt{5})$, $v_2= (\frac{1}{2}(1-\sqrt{5}),0,1)$
	\item $\lambda_3=0$, $v_3= (0,1,0)$
\end{itemize}
En este caso los autovectores están sin normalizar por lo tanto los normalizo $v_1 = \dfrac{v_1}{\norm{v_1}}$ y $v_2 = \dfrac{v_2}{\norm{v_2}}$
Llamaré $O$ a la matriz ortonormal que tiene como columnas a los autovectores de la forma bilineal y $D$ la matriz diagonal que tiene los autovalores $\lambda_i$. Entonces la forma cuadrática la puedo escribir de la siguiente manera.
\begin{equation}
\label{formula}
 F(x) = x^{t}ODO^{t}x + b^{t}OO^{t}x -1 
 \end{equation}
Calculo $O^{t}x$ que me va a dar mis nuevas coordenadas. Denotaré al vector como $(\tilde{x_1},\tilde{x_2},\tilde{x_3})$  En este caso me queda,
\[
O^tx = 
\begin{bmatrix}
\dfrac{1}{\norm{v_1}}(\frac{1}{2}(1+\sqrt{5})x_1 + x_3) \\
\dfrac{1}{\norm{v_2}}(\frac{1}{2}(1-\sqrt{5})x_1 + x_3) \\
x_2 \\
\end{bmatrix}
\]




En este caso  la parte lineal de la cuádrica nos queda lo siguiente.
\[b^tO(\tilde{x_1},\tilde{x_2},\tilde{x_3})^t = \left( \dfrac{4+2\sqrt{5}}{\norm{v_1}},\dfrac{3}{\norm{v_2}},4\right)(\tilde{x_1},\tilde{x_2},\tilde{x_3})^t \]
Expandiendo la fórmula \ref{formula} nos queda
\[ F \sim \frac{1}{2} (1+\sqrt{5}) {\ti x_1}^2 + \frac{1}{2} (1-\sqrt{5}){\ti x_2}^2 +\dfrac{4+2\sqrt{5}}{\norm{v_1}}{\ti x_1} + \dfrac{3}{\norm{v_2}}{\ti x_2} +4{\ti x_3} -1  \]

 Completando cuadrados me queda de la siguiente forma canónica. En este corresponde a una $C_{2,1}$ que es un \textit{paraboloide hiperbólico}.
 \[ F \sim \frac{1}{2} (1+\sqrt{5})x_1 ^2 +  \frac{1}{2} (1-\sqrt{5}) x_2 ^2 - 4x_3 \]
 
 Definimos la transformación ortogonal afín $f$ que manda la base que hace que la cuádrica esté en su forma canónica a la base estándar de $\mathbb R ^3$
 de la siguiente manera.
  \begin{align*}
   f\left( \dfrac{1}{\norm{v_1}}(\frac{1}{2}(1+\sqrt{5})x_1 + x_3) - \dfrac{4+2\sqrt{5}}{\norm{v_1}} \right) &= x_1  \\
   f\left( \dfrac{1}{\norm{v_2}}(\frac{1}{2}(1-\sqrt{5})x_1 + x_3) - \dfrac{3}{\norm{v_2}} \right) &= x_2  \\
   f\left( x_2 -\left( \dfrac{3}{\norm{v_2}}\right)  ^2 - \left( \dfrac{4+2\sqrt{5}}{\norm{v_1}}\right) ^2 \right)  &= x_3  \\
 \end{align*}
Pero lo que nos piden es hallar la inversa de esta transformación lineal, esto es $f^{-1}$ \footnote{A la inversa no la escribí explicitamente porque quedaba una expresión muy complicada.  Esto me hace sospechar que quizá cometí un error de cuentas que no logré encontrar.}. Ésta nos va a mandar la base canónica de $\mathbb R ^3$ en la base tal que la cuádrica tiene la forma canónica. Esta transformación afín ortogonal $g  \coloneqq f^{-1}$ hace que nos quede 
\[ G(x) \coloneqq F(g(x)) \sim \frac{1}{2} (1+\sqrt{5})x_1 ^2 +  \frac{1}{2} (1-\sqrt{5}) x_2 ^2 - 4x_3 \]


\end{document}
