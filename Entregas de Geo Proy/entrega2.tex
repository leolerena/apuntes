\documentclass[10pt,a4paper]{article}
\usepackage[latin1]{inputenc}
\usepackage{amsmath}
\usepackage{amsfonts}
\usepackage{amssymb}
\usepackage{graphicx}
\usepackage{amsmath,amsthm,amssymb}
% Include links
\usepackage{hyperref}
\usepackage{color}
\usepackage{mathtools}
\usepackage{pifont}
\usepackage{tikz-cd}
\usepackage{tcolorbox}
\usepackage{mdframed}
\usepackage{tcolorbox}
\usepackage{fancyhdr}
\pagestyle{fancy}
\lhead{Geometr�a Proyectiva}
\author{Leopoldo Lerena}


\newenvironment{exercise}[2][Ejercicio]{\begin{trivlist}
		\item[\hskip \labelsep {\bfseries #1}\hskip \labelsep {\bfseries #2.}]}{\end{trivlist}}
\newenvironment{reflection}[2][Resoluci\']{\begin{trivlist}
		\item[\hskip \labelsep {\bfseries #1}\hskip \labelsep {\bfseries #2.}]}{\end{trivlist}}


\newcommand{\norm}[1]{\left\lVert#1\right\rVert}
\newcommand{\ti}{\tilde}
\newcommand{\R}{\mathbb R}
\newcommand{\Proy}{\mathbb P ^n (k)}
\newcommand{\A}{\mathbb A}


\title{Segunda entrega.}
\date{Septiembre 2018}

\begin{document}
\maketitle


\textbf{Clasificaci�n de cu�dricas en el espacio proyectivo.}
	
	Un conjunto $Q \subset \mathbb P ^n (k)$ se llama cu�drica proyectiva si existe $F$ polinomio homog�neo de grado 2 en $n+1$ variables tal que
	\[Q = \{ [v] \in \Proy  : F(v) = 0\} \]
	
\begin{enumerate}
	\item[(a)] Si $k = \R$, probar que existe un isomorfismo proyectivo $T$ tal que $T(Q)=\mathcal C(G_{p,r})$ con $G_{p,r}$ de la siguiente forma
	\[G_{p,r} = \sum_{i=0}^{p} {X_i}^2 - \sum_{i=p+1}^{r} {X_i}^2, \hspace{0.5cm} 0\leq p \leq r \leq n, \hspace{0.5cm} r \leq 2p+1. \]
	Probar que el par $(p,r)$ no depende de la elecci�n  de $T$.
	
	\bigskip
	
	\textit{Soluci�n.}
	Como la cu�drica est� dada por un polinomio homog�neo de grado 2 en n+1 variables se corresponde a 
	\[ F(x) = \sum_{i=0}^{n} a_{ii} {X_i}^2 + \sum_{i,j}^{} 2a_{ij}X_{i} X_{j} = 0.\]
	Esto lo puedo pensar de la siguiente manera, si defino la matriz sim�trica $A = (a_{ij})_{ij}$ que tiene como entradas los coeficientes de la cu�drica
	\begin{align*}
		0 &= F(x) \\
		0 &= x^t A x. \\	
	\end{align*}
	Dado que es sim�trica y sobre los reales, puedo encontrar una matriz ortogonal $B$ tal que la diagonaliza. En particular por ser ortogonal, es inversible y por lo tanto pertenece a $Gl_{n+1}(k)$, pero tendr�a que ver que no es una homotecia para que sea un isomorfismo proyectivo. Las �nicas homotecias ortogonales son la identidad y $(-1)$ por la identidad. En ambos casos me dir�a que la matriz $A$ ya estaba diagonalizada. Por lo tanto una vez diagonalizada, si $D$ es la matriz diagonal con los elementos $d_{ii} \in \R$, pude reescribirlo como
	\begin{align*}
	0 &= F(x) \\
	0 &= x^tB^tABx \\
	0 &= x^tDx.
	\end{align*}
	Lo que acabo de hacer se puede hace id�nticamente para cu�dricas homogeneas (de tipo A) en el plano af�n que ya sabemos que son de la forma $G_{p,r}$ por lo tanto 
	esto me dice que la tengo escrita as�
	\[ G = \sum_{i=0}^{p} d_{ii}{X_i}^2 - \sum_{i=p+1}^{r} d_{ii}{X_i}^2, \hspace{0.5cm} 0\leq p \leq r \leq n, \hspace{0.5cm} r \leq 2p+1. \]
	Faltar�a que los $d_{ii}$ sean $+1$ o $-1$. Para eso necesito unas transformaciones proyectivas con matrices asociadas $S_i$ para $  0 \leq i \leq r$ tales que
	\begin{align*}
	\sum_{i=0}^{p} {X_i}^2 - \sum_{i=p+1}^{r} {X_i}^2 &= x^t{S_r}^{-1}\dots {S_0}^{-1}DS_0 \dots S_rx.
	\end{align*}
	Puedo tomar como $S_i$ la transformaci�n proyectiva con matriz asociada diagonal definida como $s_{jj} = 1$ para $j \neq i$ y $s_{ii} = \sqrt d_{ii} $ \footnote {Esto nos dice que sobre el cuerpo de los complejos todas las cu�dricas proyectivas son definidas positivas.}. Es una transformaci�n proyectiva porque en particular no es una homotecia y es inversible. \qed
	
	
	
	 
	\bigskip
	
	\item[(b)]  Sea $Q= \mathcal{C} (F)$ una cu�drica, donde
	\[F = x^2 +y^2 -z^2.\]
	Mostrar que la par�bola, la hip�rbola y la elipse son secciones de $Q$ por planos afines.
	
	\bigskip
	
	\textit{Soluci�n.} Busco planos afines tales que al intersecarlos con $Q$ obtenga una par�bola, una hip�rbola y una elipse.
	
	Para obtener una elipse considero el plano af�n que tiene los puntos con coordenada $z$ igual a $1$, lo defino como $\A_1 = \{(x,y,z) \in \Proy : z=1\} $. En tal caso la intersecci�n $Q \cap \A_1$ consiste en 
	\begin{equation*}
	 F(x) = x^2 + y^2 -1^2 = x^2 + y^2 - 1. 
	\end{equation*}
	La cu�drica que define es una elipse.
	
	Para obtener una hip�rbola considero el plano af�n que tiene los puntos con coordenada $y$ igual a $1$, de manera similar defino $\A_2 = \{(x,y,z) \in \Proy : y=1\}$. Tal que la intersecci�n $Q \cap \A_2$ consiste en
	\begin{equation*}
	F(x) = x^2 + 1^2 -z^2 = x^2 + 1^2 - z^2. 
	\end{equation*}
	La cu�drica que queda definida es una hip�rbola.
	
	Para obtener una par�bola voy a tener que considerar el plano af�n dado por los puntos que tienen coordenada $z$ igual a $1+y$. Lo defino como $\A_3 = \{(x,y,z) \in \Proy : z=1+y\}$. La intersecci�n $Q \cap \A_2$ consiste en
	\begin{align*}
	F(x) = x^2 + y^2 -(1+y)^2 &= x^2 + y^2 - y^2 + 2y - 1. \\
	F(X) &= x^2 + 2y - 1 .\\
	\end{align*}
	La cu�drica que qued� definida es una par�bola. \qed
\end{enumerate}
	
	
	



\end{document}